% "ModernCV" CV and Cover Letter
% LaTeX Template
% Version 1.11 (19/6/14)
%
% This template has been downloaded from:
% http://www.LaTeXTemplates.com
%
% Original author:
% Xavier Danaux (xdanaux@gmail.com)
%
% License:
% CC BY-NC-SA 3.0 (http://creativecommons.org/licenses/by-nc-sa/3.0/)
%
% Important note:
% This template requires the moderncv.cls and .sty files to be in the same 
% directory as this .tex file. These files provide the resume style and themes 
% used for structuring the document.
%
%%%%%%%%%%%%%%%%%%%%%%%%%%%%%%%%%%%%%%%%%

%----------------------------------------------------------------------------------------
%	PACKAGES AND OTHER DOCUMENT CONFIGURATIONS
%----------------------------------------------------------------------------------------

\documentclass[10pt,letterpaper,sans]{moderncv} % Font sizes: 10, 11, or 12; paper sizes: a4paper, letterpaper, a5paper, legalpaper, executivepaper or landscape; font families: sans or roman

\moderncvstyle{banking} % CV theme - options include: 'casual' (default), 'classic' and 'banking'
\moderncvcolor{blue} % CV color - options include: 'blue' (default), 'orange', 'grey' and 'black'

%\usepackage{lipsum} % Used for inserting dummy 'Lorem ipsum' text into the template
\usepackage[scale=0.87]{geometry} % Reduce document margins
%\setlength{\hintscolumnwidth}{3cm} % Uncomment to change the width of the dates column
%\setlength{\makecvtitlenamewidth}{10cm} % For the 'classic' style, uncomment to adjust the width of the space allocated to your name

%----------------------------------------------------------------------------------------
%	NAME AND CONTACT INFORMATION SECTION
%----------------------------------------------------------------------------------------

\firstname{Xuannan} % Your first name
\familyname{Dong} % Your last name

% All information in this block is optional, comment out any lines you don't need
\title{Resume}
\address{mailbox 5524, 4067 Miramar St.}{La Jolla, CA}
\mobile{(+1) (619) 310-3691}
\email{x3dong@ucsd.edu}

%----------------------------------------------------------------------------------------

\begin{document}

\makecvtitle % Print the CV title

\vspace{-0.3in}

%%%%%%%%%%%%%%%%%%%%%%%%%%%%%%%%%%%%%%%%%%%%%%%%%%%%%%%%%%%%%%%%%%%%%%%%%%%%%%%%%%%%%%%%
\section{Education}

\cventry{09/2021--06/2023}{M.S. in Computer Science}{University California, San Diego}{La Jolla, CA, U.S}{}{}

\cventry{08/2016--07/2020}{B.S. in Electronic Engineering(Automation) and Artificial Intelligence}{University of Science and Technology of China (USTC)}
{Hefei, Anhui, China}{}{3.61/4.3, 86.16/100. Ranking:30/140. \newline
    \cvitem{2019, 2018, 2017}{The First Prize Scholarship, School of Information Technology  (TOP 6\%)}
}

%%%%%%%%%%%%%%%%%%%%%%%%%%%%%%%%%%%%%%%%%%%%%%%%%%%%%%%%%%%%%%%%%%%%%%%%%%%%%%%%%%%%%%%%
\section{Work Experience}

\cventry{06/2022--09/2022}{WeRide}{Software Development Internship}{San Jose, CA, U.S.A}{}{}{}
\begin{itemize}
    \item[*] Researches and develops SLAM and localization technology in autonomous-driving with a focus on visual-related algorithms, including visual-based mapping and odometry.
    \item[*] Turns the algorithms into actual code running in the onboard computer and in the offline systems. Improves onboard localization performance, reliability and accuracy to make the autonomous.
    \item[*] Improves onboard localization performance, reliability and accuracy to make the autonomous vehicles safer on the road.
\end{itemize}

\cventry{02/2021--09/2021}{Intel}{Software Development Internship}{}{}{}{}
\begin{itemize}
    \item[$\bullet$] \textbf{SpeechUI App}
          \begin{itemize}
              \item[*] Developed an intelligent voice assistant web App, enabled users to order food, book tickets, navigate, etc, in an integrated platform. Built UI, frontend structure with \textbf{React}, \textbf{Redux}, and \textbf{Bootstrap}, designed dialogue processing flow to offer users prompt and accurate responses.
              \item[*] Implemented navigation part of the application. Applied Mapbox GL JS and related Plugins to support precheck for query, visualize realtime route information, intelligently offer route plan, etc.
          \end{itemize}
    \item[$\bullet$] \textbf{NLP Data Collection Platform}
          \begin{itemize}
              \item[*] Designed frontend page for an NLP Data Collection Platform. Enabled users to input sample sentences, apply different NLP models, and labeled for the processed result.
              \item[*] Built Backend service for the platform. Received and processed \textbf{RESTful} request, applied \textbf{Apache Thrift} RPC framework to call NLU/NLP service, stored data and corresponding labels into \textbf{MongoDB} database.
          \end{itemize}
\end{itemize}


\cventry{08/2020--11/2020}{Netease ThunderFire}{Data Mining Internship}{Hangzhou, Zhejiang, China}{}{}{}
\begin{itemize}
    \item[*] Collected and extracted game data with \textbf{Hadoop} and \textbf{MySQL}. Developed and maintained a Business Intelligence platform, visualized the analytic data with \textbf{Python Django} and \textbf{Vue}.
    \item[*] Utilized machine learning methods to make predictions on game trends. Worked across Finance, Player Analysis and was responsible for strengthening the connection between game players and game producers.
\end{itemize}

%%%%%%%%%%%%%%%%%%%%%%%%%%%%%%%%%%%%%%%%%%%%%%%%%%%%%%%%%%%%%%%%%%%%%%%%%%%%%%%%%%%%%%%%
\section{Course Project}

\cventry{04/2022--06/2022}{Parallel Computing Design}{C/C++}{University of California, San Diego}{}{}{}
\begin{itemize}
    \item[*]{Implemented the \textit{Aliev-Panfilov cardiac simulation} and parallelize it with \textbf{MPI}. Optimized the inner calculating loops with cache locality and vectorization. Conduct strong scaling studies and find the best parallel computing geometry.}
    \item[*]{Accelerated matrix multiplication for NVIDIA's Kepler GPU, buffered frequently accessed data in fast on-chip shared memory, applied the Memory Coalescing method. The performance reached over 400 Gflops on a matrix of size 256x256.}
\end{itemize}

\cventry{01/2022--03/2022}{Distributed File Storage System}{Golang}{University of California, San Diego}{}{}{}
\begin{itemize}
    \item[*]{Created a networked, cloud-based file storage service similar to Dropbox, users are allowed to synchronize files with "clouds". Multiple clients concurrently communicated with the service via \textbf{gRPC}.}
    \item[*]{Improved the application with the fault tolerant feature based on \textbf{RAFT} protocol, which allow a collection of machines to work as a coherent group that can survive the failures of some of its members.}
\end{itemize}

%%%%%%%%%%%%%%%%%%%%%%%%%%%%%%%%%%%%%%%%%%%%%%%%%%%%%%%%%%%%%%%%%%%%%%%%%%%%%%%%%%%%%%%%
\section{Skills}

\cvitem{Programming Language}{\textsc{C}, \textsc{JAVA}, \textsc{Python}, \textsc{MySQL}, \textsc{HTML}, \textsc{CSS}, \textsc{JavaScript}.}
\cvitem{Frameworks \& Tools}{Node.js, React.js, BootStrap, Docker, Git.}

\end{document}